%!TeX root=../main.tex
%("dica" para o editor de texto: este arquivo é parte de um documento maior)
% para saber mais: https://tex.stackexchange.com/q/78101

\chapter{Considerações finais}

Neste trabalho criamos e avaliamos um MVP de uma ferramenta de simulações interativas para o ensino de conceitos de programação para crianças do ensino fundamental. Utilizamos a metodologia de \textit{Design Science Research} proposta por \citet{wieringa2014design} para conduzir a pesquisa deste projeto, seguindo os ciclos de engenharia e \textit{design}. Dessa forma, realizamos uma investigação do problema, \textit{design} e validação do artefato, implementação e avaliação do MVP. As conclusões referentes aos resultados obtidos a partir da avaliação de usabilidade da ferramenta e do aprendizado dos alunos, bem como, as limitações do projeto e sugestões para trabalhos futuros são descritos nas seções a seguir.

\section{Conclusões}

Em relação à análise de usabilidade do MVP, concluímos que o uso de simulações interativas para ensinar conceitos de lógica de programação demonstrou potencial e despertou o interesse dos usuários iniciais que testaram a ferramenta. No geral, os alunos do 8º ano do ensino Fundamental mostraram curiosidade em explorar mais a simulação e acreditam que seus colegas poderiam aprender a usá-la muito rápido.

Principalmente para crianças que já tinham um contato anterior com programação, o MVP apresentou potencial para continuação ou complementação do aprendizados delas. A análise das respostas do formulário de usabilidade mostrou que, em média, estes alunos concordaram com as afirmações de sentimento positivo em relação à ferramenta e discordaram das afirmações de sentimento negativo. Ademais, a média da pontuação SUS obtida por meio do \textit{feedback} deste grupo de alunos foi de 71,6, o que indica uma boa usabilidade da simulação.

Para o grupo de crianças que não tinham conhecimento prévio de Computação, observamos mais concordância com as afirmações de caráter negativo. Muitos alunos alegaram que ficaram confusos ao explorar o MVP e que precisariam de ajuda para continuar a usá-lo, por exemplo. Assim, verificamos a necessidade de melhorar a maneira de introduzir os conceitos de lógica de programação para aqueles que terão um primeiro contato com esse conteúdo por meio da ferramenta. Ainda, a média do \textit{score} de usabilidade calculada através das respostas desses estudantes foi de 53,9, indicando uma usabilidade razoável da ferramenta.

Além das respostas do questionário SUS, também coletamos opiniões e sugestões livres sobre o MVP. Por meio delas, confirmamos o potencial da abordagem proposta, com relatos positivos de experiência e constatamos a demanda de algumas alterações na interatividade da simulação e apresentação do seu pseudocódigo, com o intuito de diminuir a dificuldade e confusão enfrentadas pelos alunos e de melhorar a usabilidade geral da ferramenta.

Em relação à análise da atividade de aprendizado, concluímos que, em um primeiro momento, a simulação não ajudou na compreensão dos conceitos de lógica de programação apresentados. Dentre os alunos que já tinham um conhecimento anterior em Computação, 42,9\% acertaram todas ou quase todas as questões referentes à simulação \enquote{Planejando a Festa} e apenas 14,3\% acertaram todas ou quase todas as questões sobre a simulação \enquote{Guardando os Brinquedos}. Dos alunos que não tinham contato anterior com programação, 33,3\% acertaram todas ou quase todas as questões sobre a simulação \enquote{Planejando a Festa} e a mesma quantidade em relação às questões da simulação \enquote{Guardando os Brinquedos}.

Por meio dessa análise, também notamos que os alunos sem conhecimento prévio em programação tiveram um maior número de acertos nas questões relacionadas à simulação \enquote{Guardando os Brinquedos}, com a qual eles não interagiram. Logo, caso tenha de fato ocorrido maior compreensão da \enquote{simulação} sem interações, isso poderia sugerir a necessidade de alterações e melhorias nas interações implementadas na simulação \enquote{Planejando a Festa}.

Porém, as análises referentes à atividade de aprendizado apresentam algumas ressalvas. Observamos que algumas confusões dos conceitos de programação feitas pelos alunos podem ter sido causadas pela formulação das questões da atividade. Algumas das perguntas apresentavam trechos de pseudocódigo com muitos conceitos, que inclusive se repetiam, podendo ter gerado ambiguidade no entendimento dessas perguntas. Além disso, houve limitações em relação ao tempo e finalização dessa atividade, discutidas na próxima seção.

Contudo, em outras questões, os conceitos foram apresentados de forma isolada, sendo completamente desrelacionados com algumas das respostas escolhidas, indicando a falta de compreensão do conteúdo apresentado. Dessa forma, verificamos a necessidade de reformulações para o MVP, implicando em uma nova iteração nos ciclos de \textit{design} e engenharia de DSR, o que é um passo natural para esse tipo de pesquisa.

\section{Limitações e trabalhos futuros}

Durante o projeto, um dos desafios enfrentados foi encontrar escolas para realizar a aplicação do MVP em sala de aula com alunos do Ensino Fundamental. Um dos motivos para isso foi termos esperado até o mês de Setembro para entrar em contato com as escolas, quando o planejamento das aulas já havia sido concluído. Entretanto, queríamos garantir que o MVP estivesse implementado e funcional antes de prometer uma aula para as turmas. Com a ajuda do professor Dr. Leônidas Brandão do Departamento de Ciência da Computação do IME-USP, conseguimos o contato do professor Henri Silva da Escola de Aplicação da FEUSP, que nos permitiu participar de uma de suas aulas em uma turma do 8º do Ensino Fundamental para realizar uma atividade, aplicando o MVP e coletando o \textit{feedback} dos alunos.

Outra limitação se refere ao tempo de duração da aula em que aplicamos o MVP, que foi de 50 minutos. Nesse sentido, metade da turma não conseguiu terminar de preencher o formulário que continha a atividade de aprendizado, o qual foi apresentado por último. Além disso, conforme os colegas iam terminando a atividade, observamos que algumas crianças sentiram pressa para também finalizar o preenchimento do questionário. Com isso, houve uma certa propagação das respostas entre os alunos, que compartilharam o que responderam com os demais.

Para a continuação deste trabalho sugerimos algumas alterações para o protótipo aplicado, como a introdução dos conceitos de programação na forma de tutorial antes de iniciar a interação com a simulação. O \textit{link} referente aos \enquote{Conceitos de Programação} que abre a caixa de diálogo com as definições desses poderia ser utilizado para compor esse tutorial, adicionando ainda animações e interações para dividi-lo em etapas, as quais os alunos deveriam concluir antes de explorar a simulação. Ainda, sugerimos uma melhoria na indicação dos conceito de programação na apresentação do pseudocódigo, uma vez que as cores e os nomes deles só eram visualizados ao passar o mouse em cima dos trechos de código. Ademais, recomendamos uma reformulação da atividade de aprendizado, modificando os trechos de pseudocódigo apresentado a fim de remover a ambiguidade.

Por fim, também propomos apresentar cada conceito de lógica de programação em simulações individuais, como foi proposto no projeto de Iniciação Científica (IC) intitulado ``Uma Ferramenta de Simulações Interativas para Ensino de Computação para Crianças''(\url{https://github.com/mariliatd/monografia/tree/main/ic}), realizado entre outubro de 2022 e outubro de 2023. Desse modo, os usuários poderão explorar as interações envolvidas em cada conceito separadamente, o que pode levar a melhor compreensão de cada um deles por parte das crianças e mais autonomia para explorar as simulações fora da sala de aula ou sem a ajuda de um docente.