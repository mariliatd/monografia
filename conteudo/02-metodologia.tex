%!TeX root=../tese.tex
%("dica" para o editor de texto: este arquivo é parte de um documento maior)
% para saber mais: https://tex.stackexchange.com/q/78101

\chapter{Metodologia de pesquisa} 

Para criar e avaliar o MVP proposto, seguimos a metodologia de pesquisa de \textit{Design Science}, que de forma geral envolve a elaboração de artefatos para solucionar problemas em um contexto. Neste Capítulo, explicamos o que é a \textit{Design Science Research} e como ela foi conduzida neste projeto.

\section**{\textit{Design Science Research}}

A \textit{Design Science Research} (DSR) é uma metodologia de pesquisa utilizada na elaboração de artefatos com propósitos práticos, constituindo um processo de resolução de problemas de domínio, em que o resultado deve ser avaliado pelo seu valor e utilidade. Esse paradigma é aplicado em diferentes áreas de pesquisa, como Sistemas de Informação, Gerenciamento de Negócios, Engenharia de \textit{Software}, etc, podendo ser instanciado em variantes muito diferentes \citep{dresch2015design, runeson2020design}.

\citet{hevner2004design} definiram sete diretrizes para a realização de DSR na área de Sistemas de Informação, que têm como princípio a construção e aplicação de um artefato para adquirir conhecimento sobre um problema de design e sua solução. Assim, uma pesquisa em \textit{Design Science} requer a criação de um artefato com propósito para tecnologia da informação e com caráter inovador (diretriz 1), o qual deve considerar a solução de um problema de negócio relevante (diretriz 2). Além disso, o seu design deve ser avaliado rigorosamente considerando sua utilidade, qualidade e eficácia (diretriz 3). O artefato também deve ser inovador, proporcionando contribuições claras e verificáveis para pesquisa (diretriz 4), e sua construção e avaliação devem ser feitas de forma rigorosa (diretriz 5). O processo de design do artefato deve ser conduzido como um processo de pesquisa de uma solução efetiva para um problema (diretriz 6). Por fim, os resultados da pesquisa em \textit{Design Science} devem ser comunicados de forma eficaz (diretriz 7).

%%%%%%%%%%%%%%%%%%%%%%%%%%%%%%%%%%%%%%%%%%%%%%%%%%%%%%%%%%%%%%%%%%%%%%%%%%%%%%%%
%%% Diretrizes Hevner (2004)
% \begin{enumerate}
%     \item \textbf{Design como um artefato:} criação de um artefato com propósito para TI e com caráter inovador
%     \item \textbf{Relevância do problema:} o artefato deve considerar a solução de um problema de negócios relevante
%     \item \textbf{Avaliação do design:} o design do artefato deve ser avaliado rigorosamente considerando sua utilidade, qualidade e eficácia
%     \item \textbf{Contribuições de pesquisa:} o artefato deve ser inovador, proporcionando contribuições claras e verificáveis para pesquisa
%     \item \textbf{Rigor de pesquisa:} a construção e avaliação do artefato projetado devem ser feitas de forma rigorosa
%     \item \textbf{Design como um processo de pesquisa:} o processo de design do artefato é um processo de pesquisa de uma solução efetiva para um problema
%     \item \textbf{Comunicação da pesquisa:} os resultados da pesquisa em Design Science devem ser comunicados de forma efetiva
% \end{enumerate}
%%%%%%%%%%%%%%%%%%%%%%%%%%%%%%%%%%%%%%%%%%%%%%%%%%%%%%%%%%%%%%%%%%%%%%%%%%%%%%%%

\citet{wieringa2014design} estendeu a definição de \textit{Design Science}, apresentando diretrizes para realizar pesquisas em Sistemas de Informação e Engenharia de \textit{Software}. Para o autor, a interação entre o artefato e o contexto do problema contribui para chegar à solução do problema. Dessa forma, um projeto de DS itera sobre as atividades de design e investigação. A atividade de design é decomposta em três tarefas, designadas como ciclo de design. Esse ciclo está inserido em um outro, de engenharia, no qual temos o resultado do ciclo de design sendo introduzido no contexto real e avaliado. As etapas de cada um dos ciclos estão descritas a seguir:

\noindent \textbf{Ciclo de engenharia:}

\begin{itemize}
    \item \textbf{Ciclo de design:}
    \begin{itemize}
        \item \textbf{Investigação do problema:} qual problema deve ser investigado e por quê?
        \item \textbf{Design do tratamento (\textit{Treatment design}):} projeto de um ou mais artefatos para tratar o problema.
        \item \textbf{Validação do tratamento (\textit{Treatment validation}):} análise para validar se esse projeto contribui para o tratamento do problema, caso seja implementado.
    \end{itemize}
    \item \textbf{Implementação do tratamento (\textit{Treatment implementation}):} tratamento do problema com um dos artefatos projetados.
    \item \textbf{Avaliação da implementação (\textit{Implementation evaluation}):} avaliação do sucesso do tratamento. Ao final dessa etapa, podemos ter uma nova iteração no ciclo de engenharia.
\end{itemize}

\noindent O autor também destaca que projetos de pesquisa em \textit{Design Science} estão relacionados apenas às três etapas do ciclo de \textit{design}.

\citet{runeson2020design} define etapas similares para o ciclo de engenharia, envolvendo a conceitualização do problema, o projeto da solução e a validação empírica. Os autores explicam que a \textit{Design Science} abrange duas dimensões principais: problema-solução e teoria-prática. Eles descrevem as atividades de pesquisa que são realizadas de forma iterativa através das duas dimensões, são elas:

\begin{itemize}
    \item \textbf{Conceitualização do problema:} descrição do problema.
    \item \textbf{\textit{Design} da solução:} mapeamento do problema para uma solução geral.
    \item \textbf{Abstração:} identificação de decisões de \textit{design} importantes para uma solução válida dentro de um escopo definido.
    \item \textbf{Instanciação:} implementação do artefato em contexto.
    \item \textbf{Validação empírica:} avaliação de como a solução implementada abordou o problema.
\end{itemize}

Ademais, projetos de pesquisa em \textit{Design Science} devem considerar dois fatores importantes: a relevância da pesquisa para entidades na resolução de problemas reais e o rigor para que a pesquisa seja considerada válida e confiável, contribuindo para uma determinada área. Dessa forma, a DSR é importante tanto para a produção de conhecimento científico como para a resolução de problemas reais \citep{dresch2015design, runeson2020design}.

Portanto, a \textit{Design Science} aborda problemas gerais através do estudo de instâncias específicas de problemas no contexto de pesquisa. Para realizar este Trabalho de Formatura, seguimos a metodologia de \textit{Design Science Research} em Engenharia de Software proposta por \citet{wieringa2014design}, englobando o ciclo de \textit{design} e de engenharia. Nas seções a seguir, descrevemos as etapas de cada ciclo referentes a este trabalho.

\section{Ciclo de \textit{design}}

\subsection{Investigação do problema}
% Qual problema deve ser investigado e por quê?

O problema investigado neste estudo foi o de uso de simulações interativas no ensino de conceitos de lógica de programação para crianças. Como apresentado no Capítulo~\ref{related_tools}, diferentes metodologias são adotadas para isso, utilizando recursos didáticos variados, sendo as linguagens visuais as mais utilizadas. Dessa maneira, investigamos o uso de simulações interativas para o ensino de Computação, uma vez que elas podem facilitar a visualização de ideias abstratas envolvidas nos conceitos de programação, utilizando exemplos concretos para representá-las. Ademais, essa metodologia se mostrou eficaz em outras ciências, com o uso da ferramenta PhET. Portanto, queremos expandi-la e testá-la em outros ambientes de aprendizado, como na área de Ciência da Computação.

%%%%%%%%%%%%%%%%%%%%%%%%%%%%%%%%%%%%%%%%%%%%%%%%%%%%%%%%%%%%%%%%%%%%%%%%%%%%%%%%
% Problema?
% - representações visuais dos conceitos de lógica de programação para o ensino de Computação para crianças
% - nova abordagem para ensino dos conceitos de lógica de programação para crianças
%
% Etapas
% - estudo sobre o ensino de Computação para crianças
% - busca extensiva de ferramentas de apoio pedagógico existentes
%%%%%%%%%%%%%%%%%%%%%%%%%%%%%%%%%%%%%%%%%%%%%%%%%%%%%%%%%%%%%%%%%%%%%%%%%%%%%%%%

Para investigar o problema, primeiramente, realizamos uma pesquisa sobre o estado atual do ensino de Computação para crianças em escolas no Brasil e no mundo. Em seguida, fizemos uma busca extensiva dos diversos tipos de ferramentas de apoio pedagógico existentes e similares a proposta neste trabalho, procurando entender as principais diferenças entre elas. Grande parte dessa pesquisa foi realizada durante o desenvolvimento do projeto de Iniciação Científica (IC) intitulado \enquote{Uma Ferramenta de Simulações Interativas para Ensino de Computação para Crianças}, realizado entre outubro de 2022 e outubro de 2023. O resumo dos principais resultados dessa investigação se encontra nos Capítulos \ref{introduction} e \ref{related_tools} desta monografia.

\subsection{Projeto do artefato} \label{design}
% projeto de um ou mais artefatos para tratar o problema.

O artefato projetado como tratamento do problema é um MVP de uma ferramenta de simulações interativas de conceitos de lógica de programação destinada a crianças do ensino fundamental. Um MVP (\textit{Minimum Viable Product}) ou produto mínimo viável é um produto com funcionalidades suficientes para ser utilizado por usuários iniciais que possam fornecer \textit{feedback} com o intuito de validar uma ideia. 

No ensino da Computação, alguns conceitos introdutórios são fundamentais para o aprendizado da lógica de programação, tais como variáveis, entrada e saída de dados, operadores lógicos e aritméticos, condicionais, laços de repetição, funções, vetores, matrizes, entre outros. Desse modo, para testar a viabilidade do protótipo proposto, seu projeto deve compreender um conjunto mínimo desses conceitos, a fim de verificar o entendimento deles pelos usuários. 

Assim, para projetar os artefatos ou protótipos do MVP, primeiramente, definimos alguns requisitos mínimos, listados a seguir:

\begin{itemize}
    \item O MVP deve apresentar uma tela com uma área de trabalho e uma opção de ajuda com uma documentação explicativa de uso;
    \item A área de trabalho deve apresentar um espaço fixo para a simulação e outro para o pseudocódigo correspondente;
    \item A simulação deve envolver os seguintes conceitos de lógica de programação: variáveis, entrada, saída, operadores (aritméticos, lógicos e de comparação), condicionais e laços de repetição;
    \item A simulação deve apresentar um número limitado de interações possíveis com o usuário;
    \item A simulação deve encorajar os usuários a explorá-la livremente, com controles intuitivos e uma interface que possibilite boa usabilidade;
    \item O pseudocódigo deve ser gerado automaticamente conforme as interações com a simulação vão ocorrendo.
\end{itemize}

Além disso, também definimos e descrevemos os conceitos de lógica de programação que foram simulados e a sintaxe do pseudocódigo gerado. A partir dos protótipos iniciais, estudamos ideias similares e validamos os artefatos projetados para analisar alterações necessárias e as melhores opções de implementação. O Capítulo \ref{prototypes} apresenta em detalhes o desenvolvimento dos protótipos propostos.

\subsection{Validação do artefato} \label{validation}
% análise para validar se esse projeto contribui para o tratamento do problema, caso seja implementado
A validação dos artefatos foi realizada com base nas opiniões de profissionais de ensino de Computação e matemática. Foram consultados: a professora Dra. Kelly Braghetto do Departamento de Ciência da Computação do IME, minha orientadora no trabalho de formatura, que também coordena o projeto CodificADAs USP, oferecendo cursos introdutórios de programação voltados para meninas do ensino médio; a estudante Victoria Nóvoa, do curso de Licenciatura em Educomunicação da USP, instrutora de tecnologia educacional de crianças no Colégio Santa Cruz; e o professor Henri Silva da Escola de Aplicação da Faculdade de Educação da USP, que leciona matemática para alunos do 8° ano do Ensino Fundamental II.

Em conversas com os avaliadores, foi possível obter \textit{feedbacks} para verificar a usabilidade e utilidade do MVP a partir dos protótipos projetados, possibilitando o refinamento deles e a escolha de um artefato para implementação, além da definição do público alvo específico para testar a aplicação. No Capítulo \ref{prototypes}, descrevemos o processo de validação e as alterações realizadas. 

\section{Ciclo de engenharia}

Após a execução das etapas anteriores, completando uma iteração no ciclo de \textit{design}, realizamos os demais passos do ciclo de engenharia, descritos a seguir.

\subsection{Implementação da simulação}
% tratamento do problema com um dos artefatos projetados
O MVP da simulação foi desenvolvido utilizando o \textit{framework} de código aberto Vue.js\footnote{\url{https://vuejs.org/}}, que é muito utilizado para criar aplicações de página única (\textit{single-page applications}), com a intenção de agilizar o desenvolvimento, permitindo a utilização de soluções existentes e reduzindo a necessidade de escrever códigos do zero. O projeto criado com o \textit{framework} foi inicializado para ser utilizado com a linguagem de programação Typescript. Em conjunto com o Vue.js, utilizamos o Vuetify\footnote{\url{https://vuetifyjs.com/en/}}, um \textit{framework} de componentes de UI de código aberto, que contém diversos \textit{layouts} prontos e componentes dinâmicos.

O código desenvolvido ao longo do projeto, com a implementação do artefato, está disponível no seguinte repositório: \url{https://github.com/mariliatd/logic-sims-mvp}, sob a licença MPL-2.0 (\textit{Mozilla Public License Version 2.0}). Para desenvolver o MVP, utilizamos a organização em componentes que o \textit{framework} Vue possibilita. Dessa maneira, foi possível definir cada conceito de lógica de programação como um componente isolado a ser reutilizado dentro de um \textit{template} de simulação, conforme necessário. Também separamos os componentes referentes aos elementos visuais da simulação do pseudocódigo, criando maior modularidade no código.

Assim, com o artefato implementado, criamos uma página para disponibilizar o MVP, através da ferramenta de versionamento de códigos GitHub, para a aplicação e avaliação da ferramenta em sala de aula.

\subsection{Avaliação da simulação}
% avaliação do sucesso do tratamento
A avaliação do MPV da ferramenta de simulação foi realizada utilizando um formulário de usabilidade e uma atividade para verificação do aprendizado dos conceitos de lógica de programação. Primeiramente, por meio do questionário de usabilidade, métrica comum na avaliação de um protótipo ou sistema, o MVP foi analisado considerando a facilidade de uso, de aprendizado e satisfação. Em particular, utilizamos o questionário SUS (\textit{System Usability Scale}), aplicado aos usuários finais.

Apesar de não haver medidas absolutas de usabilidade, é possível utilizar escalas gerais para comparar usabilidade em determinados contextos. O SUS representa uma escala de usabilidade com 10 itens que pode ser utilizada para avaliar sistemas \citep{brooke1996sus}. Ele é baseado na escala Likert, a qual contém afirmações e os avaliadores indicam o grau de concordância ou discordância com cada uma, que varia de 1 a 5. Recomenda-se que as respostas para cada item sejam registradas de imediato, sem que os respondentes levem muito tempo pensando nelas.

Para obter o \textit{score} de usabilidade do sistema a partir do questionário SUS, primeiramente, a contribuição de cada item é normalizada para uma escala de 0 a 4. Em seguida, a soma das pontuações dos itens é multiplicada por 2.5, obtendo um \textit{score} em uma escala de 0 a 100. Entretanto, os autores não revelaram como analisar esta pontuação.

\citet{bangor2008empirical, bangor2009determining} realizaram um estudo onde avaliaram a usabilidade de diversos produtos e serviços utilizando o questionário SUS. Eles analisaram a média de pontuação do SUS em 273 estudos com cerca de 3.500 pesquisas individuais. Os autores realizaram uma interpretação dessa pontuação, comparando os quartis dos \textit{scores}. Além disso, eles adicionaram uma afirmação ao questionário para avaliar a usabilidade do sistema através de uma escala de classificação de adjetivos, obtendo uma correlação entre eles e a média das pontuações. As médias que correspondem a cada adjetivo são as seguintes:

\begin{enumerate}
    \item pior imaginável (\textit{worst imaginable}): 12.5
    \item horrível (\textit{awful}) : 20.3
    \item ruim (\textit{poor}): 35.7
    \item ok (\textit{ok}): 50.9
    \item bom (\textit{good}): 71.4
    \item excelente (\textit{excellent}): 85.5
    \item melhor imaginável (\textit{best imaginable}): 90.9
\end{enumerate}

Os autores ainda expressaram preocupações com o uso do adjetivo \enquote{ok}, uma vez que ele sugere uma experiência aceitável com o sistema, enquanto a média de pontuações na faixa dele sugere deficiências. Eles acreditam que o termo \enquote{razoável} (\textit{fair}) seria melhor indicativo da usabilidade percebida pelos usuários.

Ademais, o SUS é considerado um questionário robusto e confiável, tendo sido utilizado em diversos projetos de pesquisa e avaliações na indústria \citep{brooke1996sus}. Embora não tenha sido projetado considerando necessidades específicas de compreensão por crianças, o questionário tem sido utilizado em vários estudos de testes e avaliações de ferramentas com crianças de diversas faixas etárias com sucesso \citep[i.e.][]{wronska2015ipad, dexheimer2017usability, sanchez2020usability, tasfia2023evaluating}. 

\citet{putnam2020adaptation} realizaram uma adaptação e teste do SUS com crianças na faixa etária de 7 a 11 anos. Os autores adaptaram o questionário, com o auxílio de professores da educação básica, em um contexto de aplicativos móveis de jogos focados no ensino de programação e pensamento computacional. As afirmações foram ainda modificadas pensando na separação de dois grupos de faixa etária, entre 7 e 8 anos e entre 9 e 11 anos. A Tabela \ref{table:sus} mostra os enunciados do SUS original e as adaptações propostas para os dois grupos mencionados, em inglês.  
Além das simplificações das afirmações, os autores também utilizaram uma representação visual da escala de Likert (Figura \ref{figure:likert}), sugerida pelos docentes participantes do experimento.
Os resultados obtidos nos experimentos mostraram que o questionário modificado juntamente com a escala visual foi compreendido pelas crianças participantes, necessitando apenas de clarificações mínimas. Eles ainda sugeriram alterações nas afirmações 6, 8 e 10 para melhorar a compreensão e a confiabilidade delas.


\begin{table}[h!]
\centering
\resizebox{\textwidth}{!}{%
\begin{tabular}{|c|c|c|c|}
\hline
\textbf{Afirmativa} & \textbf{SUS original}                                                                                                                 & \textbf{SUS adaptado: Grupo 9-11 anos}                                                                                                   & \textbf{SUS adaptado: Grupo 7-8 anos}                          \\ \hline
\textbf{1}           & \begin{tabular}[c]{@{}c@{}}I think that I would like to use this\\ system frequently.\end{tabular}                                    & \begin{tabular}[c]{@{}c@{}}If I had this {[}app{]} on my iPad, \\ I think that I would like to play it a lot.\end{tabular} & I would like to play {[}app{]} a lot more.       \\ \hline
\textbf{2}           & I found the system unnecessary complex.                                                                                               & \begin{tabular}[c]{@{}c@{}}I was confused many times \\ when I was playing {[}app{]}.\end{tabular}                         & {[}app{]} was hard to play.                      \\ \hline
\textbf{3}           & I thought the system was easy to use.                                                                                                 & I thought {[}app{]} was easy to use.                                                                                       & I thought {[}app{]} was easy to use.             \\ \hline
\textbf{4}           & \begin{tabular}[c]{@{}c@{}}I think that I would need the support of a\\  technical person to be able to use this system.\end{tabular} & \begin{tabular}[c]{@{}c@{}}I would need help from an\\  adult to continue to play {[}app{]}.\end{tabular}                  & I would need help to play {[}app{]} more.        \\ \hline
\textbf{5}           & \begin{tabular}[c]{@{}c@{}}I found the various functions in this\\ system were well integrated.\end{tabular}                          & \begin{tabular}[c]{@{}c@{}}I always felt like I knew what to do\\  next when I played {[}app{]}.\end{tabular}              & I knew what to do next when I played {[}app{]}.  \\ \hline
\textbf{6}           & \begin{tabular}[c]{@{}c@{}}I thought there was too much\\ inconsistency in the system.\end{tabular}                                   & \begin{tabular}[c]{@{}c@{}}Some of the things I had to do when\\  playing {[}app{]} did not make sense.\end{tabular}       & Some things in {[}app{]} made no sense.          \\ \hline
\textbf{7}           & \begin{tabular}[c]{@{}c@{}}I would imagine that most people would\\ learn to use this system very quickly.\end{tabular}               & \begin{tabular}[c]{@{}c@{}}I think most of my friends could\\  learn to play {[}app{]} very quickly.\end{tabular}          & {[}app{]} would be easy for my friends to learn. \\ \hline
\textbf{8}           & I felt the system was cumbersome to use.                                                                                              & \begin{tabular}[c]{@{}c@{}}Some of the things I had to do \\ to play {[}app{]} were kind of weird.\end{tabular}            & To play {[}app{]} I had to do some weird things. \\ \hline
\textbf{9}           & I felt very confident using the system.                                                                                               & I was confident when I was playing {[}app{]}.                                                                              & I was proud of how I played {[}app{]}.           \\ \hline
\textbf{10}          & \begin{tabular}[c]{@{}c@{}}I needed to learn a lot of things before\\ I could get going with this system.\end{tabular}                & \begin{tabular}[c]{@{}c@{}}I had to learn a lot of things \\ before playing {[}app{]} well.\end{tabular}                   & There was a lot to learn to play {[}app{]}.      \\ \hline
\end{tabular}%
}
\caption{Afirmativas em inglês do questionário de avaliação de usabilidade de sistemas SUS e adaptações propostas por \citet{putnam2020adaptation} para crianças entre 7 e 11 anos, dividida em dois grupos.}
\label{table:sus}
\end{table} 

\begin{figure}[h!]
    \centering
    \setlength{\fboxrule}{0.1pt} % espessura da borda da figura
    \fbox{\includegraphics[scale=0.5]{escalaLikertVisual.png}}
    \caption{Representação visual da escala de Likert \citep{putnam2020adaptation}.}
    \label{figure:likert}
\end{figure}


Dessa forma, utilizamos uma adaptação do questionário SUS para obter o \textit{feedback} dos estudantes sobre o MVP para avaliá-lo em relação a sua usabilidade.  
Ademais, através da atividade de aprendizado dos conceitos de programação, analisamos o potencial do ensino de Computação utilizando simulações interativas. No Capítulo \ref{evaluation}, descrevemos a atividade realizada em sala de aula com a aplicação dos dois formulários para avaliação do artefato implementado.
