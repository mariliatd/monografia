%!TeX root=../tese.tex
%("dica" para o editor de texto: este arquivo é parte de um documento maior)
% para saber mais: https://tex.stackexchange.com/q/78101

% As palavras-chave são obrigatórias, em português e em inglês, e devem ser
% definidas antes do resumo/abstract. Acrescente quantas forem necessárias.
\palavraschave{simulações interativas, ensino de Computação, lógica de programação, MVP}

\keywords{interactive simulations, computer teaching, programming logic, MVP}

% O resumo é obrigatório, em português e inglês. Estes comandos também
% geram automaticamente a referência para o próprio documento, conforme
% as normas sugeridas da USP.
\resumo{
O ensino de Computação para crianças e adolescentes é cada vez mais importante na Era Digital. O conhecimento de programação promove muitas competências fundamentais para o pensamento crítico e lógico desses jovens. Por isso, ele vem ganhando cada vez mais espaço nas escolas de educação básica brasileiras, tornando-se parte dos currículos escolares. Ferramentas de apoio pedagógico são fundamentais nesse processo de aprendizagem e o uso de simulações interativas na educação tem se mostrado eficaz em outras ciências. Nesse sentido, este Trabalho de Formatura teve como objetivo a criação e validação de um produto mínimo viável (MVP - \textit{Minimum Viable Product}) de uma ferramenta \textit{web} de simulações interativas de conceitos de lógica de programação para o ensino de Computação no ensino fundamental. A ferramenta considera conceitos de lógica de programação básicos, como variáveis, entrada e saída de dados, condicionais e laços de repetição, permitindo que os estudantes a explore livremente. Ademais, ela apresenta o pseudocódigo correspondente, a fim de promover o contato com a estrutura real de um código. Para realizar este trabalho, seguimos a metodologia \textit{Design Science Research} (DSR) em Engenharia de Software, para construção de artefatos em contexto. Assim, investigamos o problema por meio de uma contextualização e, a partir disso, projetamos e validamos um artefato, o MVP. Também realizamos a implementação dele utilizando o \textit{framework} Vue.js em Typescript. Por fim, avaliamos a usabilidade do MVP proposto e o aprendizado dos alunos através dele, em uma atividade com estudantes do ensino fundamental, por meio de um questionário de usabilidade (SUS - \textit{System Usability Scale}) e de um questionário sobre o aprendizado dos conceitos de lógica de programação. A análise dos resultados dessas avaliações nos permitiu concluir que o uso de simulações interativas para ensino de Computação demonstrou potencial, principalmente quando aplicadas a crianças que já tiveram um contato anterior com programação, apresentando uma pontuação de usabilidade considerada boa. Considerando alunos que não têm conhecimento prévio em Computação, a usabilidade da ferramenta foi classificada como razoável segundo a pontuação do SUS e observamos a necessidade de melhorias na forma de introduzir os conceitos para esses usuários. Por meio da atividade de aprendizado, concluímos que a abordagem proposta não ajudou na compreensão dos conceitos de programação, uma vez que a maioria dos alunos não conseguiram identificá-los em trechos de pseudocódigo no questionário. Entretanto, observamos que algumas das confusões feitas pelos estudantes podem ter sido causadas pela formulação das questões avaliativas. Dessa maneira, foi possível notar a necessidade de melhorias na simulação e no pseudocódigo do MVP avaliado, implicando em uma nova interação nos ciclos de design e engenharia de DSR, o que seria um passo natural para este tipo de pesquisa.
}

\abstract{
Teaching computer science to children and teenagers is increasingly important in the Digital Age. Programming knowledge promotes many fundamental skills for critical and logical thinking in these young people. Therefore, it gained more and more space in Brazilian basic education schools, becoming part of the school curricula. Pedagogical support tools are fundamental in this learning process and the use of interactive simulations in education has proven effective in other sciences. In this sense, this capstone project aimed to create and validate a minimum viable product (MVP) of a web tool for interactive simulations of programming logic concepts for teaching computing in elementary school. The tool considers basic programming logic concepts, such as variables, data input and output, conditionals, and loops, allowing students to explore it freely. Furthermore, it presents the corresponding pseudocode, to promote contact with the real structure of a code. To carry out this work, we followed the Design Science Research (DSR) methodology in Software Engineering, for building artifacts in context. Thus, we investigated the problem through contextualization and, based on that, we designed and validated an artifact, the MVP. We also implemented it using the Vue.js framework in Typescript. Finally, we evaluated the usability of the proposed MVP and the students' learning through it in an activity with elementary school students applying a usability questionnaire (SUS - System Usability Scale) and a questionnaire about the learning of programming logic concepts. Analysis of the results of these evaluations allowed us to conclude that using interactive simulations for teaching computing demonstrated potential, especially when applied to children who have had previous contact with programming, presenting a usability score considered good. Considering students who have no prior knowledge of computing, the tool's usability was classified as fair according to the SUS score and we observed the need for improvements in the way concepts are introduced to these users. Through the learning activity, we concluded that the proposed approach did not help in understanding programming concepts, since most students could not identify them in pseudocode snippets in the questionnaire. However, we observed that the formulation of the assessment questions may have caused some confusion among students. In this way, it was possible to note the need for improvements in the simulation and pseudocode of the evaluated MVP, implying a new interaction in the DSR design and engineering cycles, which would be a natural step for this type of research. 
}
